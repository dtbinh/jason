\documentclass{article}
\usepackage{graphicx}
\usepackage{html}
\usepackage{hthtml}


\begin{document}

\title{Jason and EIS integration \\\small{(release 0.3)}}
%\author{Jomi F. H\"ubner}
\maketitle

\begin{abstract}
  This document describes (i) how to use environments developed as
  defined by the Environment Interface Standard
  (\htlink{EIS}{http://cig.in.tu-clausthal.de/eis}) in
  \htlink{Jason}{http://jason.sf.net} projects; and (ii) how Jason
  environment can be adapted to be used as EIS environment.
\end{abstract}

\tableofcontents


\section*{Using EIS environments in Jason}

Jason projects can use EIS environments by means of  the
EISAdapter. EISAdapter is a Jason environment, i.e. it follows all
requirements of an environment for Jason projects, that delegates
perception and action to an EIS environment. It basically maps agents
actions to EIS actions and gives EIS perception back to Jason agents.


\subsection*{Jason Project Definition}

EIS has the concept of \emph{entities} which are associated to agents
that can then control those entities.  Since Jason does not have this
concept, we propose two solutions:
\begin{itemize}
\item The relation agent-entity is very simple: each agent has exactly
  one entity and their names are the same. In this case, the Jason
  designer does not need to specify the relation.
\item The designer defines all the relations in the project.
\end{itemize}

The Carriage example is used in the sequence to illustrate the two
solutions (see EIS documentation for more details about this
environment). 

Using the first solution, the Jason project for the Carriage environment is:
\begin{verbatim}
MAS demo2ag {
   infrastructure: Centralised

   environment: jason.eis.EISAdapter(
                       "lib/eis-0.2-carriage.jar" // jar file containing the 
                                                  // environment implementation
                )

   agents:
       robot1;
       robot2;
}
\end{verbatim}
Notice that the environment class is
\texttt{jason.eis.EISAdapter}. This class loads the EIS environment
provided in the jar file and delegates action/perception to
it. Regarding entities, each agent will control an entity that has the
same name as it, i.e. agent robot1 controls entity robot1 and agent
robot2 controls entity robot2.

Using the second solution, the Jason project for the Carriage environment is:
\begin{verbatim}
MAS demo1ag {
   infrastructure: Centralised

   environment: jason.eis.EISAdapter(
                      "lib/eis-0.2-carriage.jar" // jar file containing the 
                                                 // environment implementation
                      agent_entity(robot,robot1), 
                      agent_entity(robot,robot2) // agent x entities relation  
                )

   agents:
       robot; 
}
\end{verbatim}
In this case, the entities associated to each agent are explicitly
defined by pairs <agent>, <entity>. In the above example, agent robot
will control both entities of the environment.

If parameters in the form ``id = value'' need to be used, the map
structure have to be used, for instance:
\begin{verbatim}
MAS wumpus {

    infrastructure: Centralised

    environment: jason.eis.EISAdapter(
                         "lib/wumpusenv.jar",    // EIS environment
                         agent_entity(ag,agent), // entities
                         map(file,"maps/wumpus.kt5.wld")) // initialisation parameters	
    agents:
        ag;
}
\end{verbatim}

\subsection*{Agent programming}

The agents are programmed as usual. When the agent needs to handle
entities, two solutions are implemented.

\begin{description}
\item[perception] every perception given by the EIS environment is
  annotated with the entity that has produced the perception. E.g.,
  the perception in the Carriage example is:
\begin{verbatim}
carriagePos(2)[entity(robot2),source(percept)]
\end{verbatim}

\item[action] actions are sent to all entities by default, e.g.:
\begin{verbatim}
... push; ...
\end{verbatim}

  When the programmer wants to define a target entity he/she has to
  use a special action \texttt{ae} with to arguments: the action and
  the entity (as string). E.g.:
\begin{verbatim}
ae(push,"robot1")
\end{verbatim}
\end{description}

As examples, it follows the code of the three agents used in the two
projects cited in the previous section.

\subsection*{Code of robot1.asl in project demo2ag}
\begin{verbatim}
!start.       // initial goal

+!start       // plan to achieve the goal 
  <- push;    // act on its entity
     !!start. // continue doing the same
\end{verbatim}
Notice that actions are used as usual in Jason.


\subsection*{Code of robot2.asl in project demo2ag}
\begin{verbatim}
!start.                          // initial goal

+!start 
  <- wait; 
     push; 
     !!start.

+step(X) : carriagePos(C) 
  <- .print("Step ",X,", carriage at ",C).
\end{verbatim}

\subsection*{Code of robot.asl in project demo1ag}
\begin{verbatim}
!start1.  // this agent has two intentions, one for each entity
!start2.


+!start1 <- ae(push,"robot1"); !!start1.                    // act on entity robot1
+!start2 <- ae(wait,"robot2"); ae(push,"robot2"); !!start2. // act on entity robot2

+step(X) : carriagePos(C) 
  <- .print("Step ",X,", carriage at ",C).
\end{verbatim}

\subsection*{How to create a project with Jason-EIS integration}
\begin{enumerate}
\item Copy eis.jar, eis-jason.jar and the environment .jar files to the lib directory of your project.
\item Define the environment of Jason project as described in the second section of this document.
\end{itemize}


\section*{EISifying Jason environments}

Jason Environments (those defined to be included in Jason Projects)
can be used as an EIS environment quite easily. Only entities have to
be defined, since they do not belong to the Jason concept of environment. 

The domestic-robot example is used to illustrate the process. All the
code and more details are included in the JasonEIS 0.3
(\url{../examples/domestic-robot/doc/eis-doc.txt}).

\begin{center}
\begin{rawhtml}
  <IMG WIDTH="800" SRC="../examples/domestic-robot/doc/overview.png" />
\end{rawhtml}  
\end{center}

The following steps describes how to ``export'' a Jason Environment to EIS:
\begin{enumerate}
\item Create a new class that extends the Jason Adapter for EIS
  environment. In its constructor instantiate the Jason Environment
  class and define the entities:
\begin{verbatim}
public class EISHouseEnv extends JasonAdapter {

    public EISHouseEnv() {
        jasonEnv = new HouseEnv(); // create the instance of Jason Environment
        try {
            addEntity("robot");
            addEntity("owner");
            addEntity("supermarket");
        } catch (EntityException e) {
            e.printStackTrace();
        }
    }
       
}
\end{verbatim}
  The JasonAdapter class will be in charge to map perception and
  action between Jason and EIS. It also maps the INIT command to the
  init method in Jason environments and the KILL command to the stop
  method. (The other commands are ignored since they do not have
  counter part in Jason.)

%   Note that Jason Environments suppose that the init method is always
%   called; while EIS defines INIT command as optional. The above
%   solution guarantees that init will be called, but it can be called
%   twice (in the constructor and by the INIT command). With that
%   solution the Jason environment have to be prepared to have its init
%   method called twice.


\item Create a jar file that includes your new class, the Jason
  classes, and the Jason-EIS classes. For instance (from build.xml of
  domestic-robot):
\begin{verbatim}
  <target name="jar" depends="compile">
        <delete file="${ant.project.name}.jar" />
        <copy file="${jasonJar}" tofile="${ant.project.name}.jar" />
        <jar update="yes" jarfile="${ant.project.name}.jar" >
            <fileset dir="${build.dir}">
                <include name="**/*.class" />
            </fileset>
            <fileset dir="${jason-eis-dir}/bin/classes">
                <include name="jason/**/*.class" />
            </fileset>
            <manifest>
                 <attribute name="Main-Class" value="EISHouseEnv"/>
            </manifest>
        </jar>      
    </target>
\end{verbatim}
\end{enumerate}



\section*{Data Translation}

Data types in EIS and Jason are different, they are so translated as
defined in the following table.

\begin{tabular}{cc}
  \emph{Type in EIS} & \emph{Type in Jason} \\ \hline
  Identifier & Term \\
  Numeral  & NumberTerm \\
  ParameterList & ListTerm \\
  Function & Literal
\end{tabular}

\end{enumerate}
